\subsection{Программная реализация на языке Python}

Для реализации распределения Эрланга использовался язык программирования Python. Алгоритм заключается в том, что для каждого элемента выборки генерируется сумма \( k \) случайных величин, каждая из которых распределена по экспоненциальному закону с параметром \( \lambda \). Данная сумма соответствует распределению Эрланга с параметрами \( k \) и \( \lambda \).

\begin{lstlisting}
import math
import random

k = 2
lambda_param = 0.0144

def generate_exponential(lambda_param):
    return -math.log(random.random()) / lambda_param

def generate_erlang(k, lambda_param, size=1000):
    erlang_samples = []
    for _ in range(size):
        sample = sum(
            generate_exponential(lambda_param) for _ in range(k)
        )
        erlang_samples.append(sample)
    return erlang_samples

generated_data = generate_erlang(k, lambda_param)
\end{lstlisting}

Генерация данных проводится с помощью функции `generate\_erlang`, которая для каждой случайной величины просуммирует \( k \) экспоненциальных случайных величин, сгенерированных с параметром \( \lambda \).

\subsubsection{Вывод}

Алгоритм генерации случайных величин по закону Эрланга был успешно реализован, что позволяет получать величины, соответствующие этому распределению, для дальнейшего анализа.
