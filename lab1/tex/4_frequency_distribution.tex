\subsection{График гистограммы распределения частот}

\FloatBarrier
\begin{figure}[h]
	\centering
	\includegraphics[width=0.7\textwidth]{../data/histogram.png}
	\caption{Гистограмма распределения частот заданной ЧП}
\end{figure}
\FloatBarrier

\subsection{Выводы}

Построенная гистограмма (рис. 1) представляет распределение частот для числовой последовательности. На графике можно заметить, что большинство значений сосредоточены в области с малыми числами в районе сотни, а частота встречаемости значений резко снижается по мере увеличения значений или уменьшения от супремума в этой точке. То есть нельзя сделать вывод о экспоненциальном распределении, а также и о нормальном распределении

Такой характер распределения данных указывает на \textbf{распределение Эрланга}. 
