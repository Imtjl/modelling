В данной работе рассматривается случайная числовая последовательность (ЧП), для которой требуется рассчитать основные числовые моменты. Что значит \textit{случайная}? Значит состоит из случайных величин, то есть набора значений, полученных в результате наблюдений или экспериментов. Числовые моменты используются для описания характеристик распределения таких данных.

В этой работе для различных выборок случайной величины (10, 20, 50, 100, 200 и 300 значений) были рассчитаны следующие числовые моменты:
\begin{itemize}
	\item математическое ожидание;
	\item дисперсия;
	\item среднеквадратическое отклонение;
	\item коэффициент вариации;
	\item доверительные интервалы.
\end{itemize}

\subsection{Математическое ожидание}
\textbf{Математическое ожидание} — это среднее значение случайной величины, которое показывает её центр распределения. Это важный числовой момент, который описывает центр масс распределения данных, если простыми словами -- какая величина наиболее ``ожидаема`` в выборке.

Математическое ожидание для выборки из $n$ значений вычисляется по следующей формуле:
\[
	\mu = \frac{1}{n} \sum_{i=1}^{n} x_i
\]
где $n$ — количество элементов в выборке, $x_i$ — значение $i$-го элемента.


\subsection{Дисперсия}
\textbf{Дисперсия} показывает степень разброса значений относительно математического ожидания. Она позволяет оценить, насколько сильно данные отклоняются от среднего значения.

Дисперсия для выборки вычисляется по формуле:
\[
	\sigma^2 = \frac{1}{n - 1} \sum_{i=1}^{n} (x_i - \mu)^2
\]
где $\mu$ — математическое ожидание, $n$ — количество элементов выборки.

В этой работе дисперсия рассчитывается для каждой подвыборки, чтобы оценить, как изменяется разброс значений по мере увеличения объёма выборки.

\subsection{Среднеквадратическое отклонение}
\textbf{Среднеквадратическое отклонение (СКО)} — это квадратный корень из дисперсии. Оно показывает стандартную величину отклонения значений от их среднего. Обратите внимание, я сказал ``стандартную`` величины, потому что этот показатель также называется стандартным отклонением.

СКО вычисляется по формуле:
\[
	\sigma = \sqrt{\sigma^2}
\]
Вопрос, а зачем нужен СКО если дисперсия уже показывает вариабельность данных? Потому что он позволяет оценить отклонение значений от среднего \textit{в тех же} единицах измерения, что и исходные данные, что делает его более интуитивно понятным.

\subsection{Коэффициент вариации}
\textbf{Коэффициент вариации} представляет собой отношение стандартного отклонения к математическому ожиданию и выражается в процентах. Этот показатель используется для оценки относительного разброса данных и полезен при сравнении данных с разными средними значениями.

Формула коэффициента вариации:
\[
	CV = \frac{\sigma}{\mu} \times 100\%
\]

\subsection{Доверительные интервалы}
\textbf{Доверительный интервал} — это диапазон значений, в который с определённой вероятностью попадает истинное математическое ожидание. В данной работе использовались доверительные интервалы для уровней доверия 0.9, 0.95 и 0.99.

Формула для расчёта доверительного интервала:
\[
	CI = \mu \pm t_{\alpha/2} \cdot \frac{\sigma}{\sqrt{n}}
\]
где $t_{\alpha/2}$ — критическое значение $t$-распределения для заданного уровня доверия, $\sigma$ — стандартное отклонение, $n$ — объём выборки.

Чем больше объём выборки, тем уже доверительный интервал, что означает более точную оценку математического ожидания.

\subsection{Форма 1: Таблица числовых моментов для различных выборок}
В таблице ниже представлены числовые моменты для выборок из 10, 20, 50, 100, 200 и 300 значений. Для каждой подвыборки вычислены математическое ожидание, дисперсия, среднеквадратическое отклонение, коэффициент вариации и доверительные интервалы для различных уровней доверия, а также относительные отклонения рассчитанных значений от величин, полученных для выборки из 300 значений.

\begin{table}[h]
	\centering
	\resizebox{\textwidth}{!}{
		\begin{tabular}{|c|c|c|c|c|c|c|c|}
			\hline
			\multirow{2}{*}{\textbf{Характеристика}}   &      & \multicolumn{6}{c|}{\textbf{Количество случайных величин}}                                                                                        \\
			\cline{2-8}
			                                           &      & \textbf{10}                                                & \textbf{20} & \textbf{50} & \textbf{100} & \textbf{200} & \textbf{300}               \\
			\hline
			\multirow{2}{*}{\textbf{Мат. ожидание}}    & Знач & 147.476                                                    & 148.874     & 162.476     & 181.411      & 178.538      & \multirow{2}{*}{175.513}   \\
			\cline{2-7}
			                                           & $\%$ & -15.974                                                    & -15.178     & -7.428      & 3.360        & 1.723        &                            \\
			\hline
			\multirow{2}{*}{\textbf{Дов. инт. (0.9)}}  & Знач & \pm28.544                                                  & \pm26.761   & \pm23.416   & \pm19.893    & \pm14.512    & \multirow{2}{*}{\pm11.701} \\
			\cline{2-7}
			                                           & $\%$ & \pm143.937                                                 & \pm128.699  & \pm100.119  & \pm70.008    & \pm24.018    &                            \\
			\hline
			\multirow{2}{*}{\textbf{Дов. инт. (0.95)}} & Знач & \pm35.224                                                  & \pm32.392   & \pm28.068   & \pm23.773    & \pm17.316    & \multirow{2}{*}{\pm13.956} \\
			\cline{2-7}
			                                           & $\%$ & \pm152.392                                                 & \pm132.100  & \pm101.113  & \pm70.339    & \pm24.077    &                            \\
			\hline
			\multirow{2}{*}{\textbf{Дов. инт. (0.99)}} & Знач & \pm 50.604                                                 & \pm44.277   & \pm37.431   & \pm31.467    & \pm22.838    & \multirow{2}{*}{\pm18.385} \\
			\cline{2-7}
			                                           & $\%$ & \pm175.250                                                 & \pm140.836  & \pm103.599  & \pm71.159    & \pm24.225    &                            \\
			\hline
			\multirow{2}{*}{\textbf{Дисперсия}}        & Знач & 2424.613                                                   & 4790.369    & 9753.900    & 14354.424    & 15422.509    & \multirow{2}{*}{15088.212} \\
			\cline{2-7}
			                                           & $\%$ & -83.930                                                    & -68.251     & -35.354     & -4.863       & 2.216        &                            \\
			\hline
			\multirow{2}{*}{\textbf{Ср. кв. о.}}       & Знач & 49.240                                                     & 69.212      & 98.762      & 119.810      & 124.187      & \multirow{2}{*}{122.834}   \\
			\cline{2-7}
			                                           & $\%$ & -59.913                                                    & -43.654     & -19.597     & -2.462       & 1.102        &                            \\
			\hline
			\multirow{2}{*}{\textbf{Коэф. вариации}}   & Знач & 33.389\%                                                   & 46.491\%    & 60.786\%    & 66.043\%     & 69.558\%     & \multirow{2}{*}{69.986}    \\
			\cline{2-7}
			                                           & $\%$ & -52.292                                                    & -33.571     & -13.146     & -5.633       & -0.611       &                            \\
			\hline
		\end{tabular}}
	\caption{Числовые моменты для различных выборок ЧП}
\end{table}

\subsection{Вывод по 1 разделу}
Проведя рассчёт числовых моментов заданной числовой последовательности можно сделать однозначный вывод о том, что при росте количества случайных величин (т.е. объёма выборки) -- отклонения рассчитанных значений относительно самой большой выборки в 300 элементов \textbf{уменьшаются} (по модулю), как уменьшается погрешность полученных величин и точность эксперимента соответственно.
