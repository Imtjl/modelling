% Section 3: Таблица измерений
\section{Заполненная таблица 1.1}

% Формулы для расчётов
\subsection{Формулы для расчёта}

Ток через нагрузку рассчитывается по формуле:
\[
	I_n = \frac{U_n}{R_n} \; \text{[А]},
\]
где \( U_n \) — измеренное напряжение на нагрузке, а \( R_n \) — сопротивление нагрузки.

Абсолютная погрешность тока:
\[
	\Delta I_n = \frac{\Delta U_n}{R_n}.
\]
, где R - известная константа, а U - измеренная величина

Внутреннее сопротивление источника для каждого промежутка между измерениями рассчитывается по формуле:
\[
	r_k = \frac{U_{n_k} - U_{n_{k+1}}}{I_{n_{k+1}} - I_{n_k}} \; \text{[Ом]}.
\]

Абсолютная погрешность внутреннего сопротивления \( \Delta r_k \):
\[
	\Delta r_k = \frac{|\Delta(U_{n_k} - U_{n_{k+1}}) \cdot (I_{n_{k+1}} - I_{n_k}) + \Delta(I_{n_{k+1}} - I_{n_k}) \cdot (U_{n_k} - U_{n_{k+1}})|}{(I_{n_{k+1}} - I_{n_k})^2}.
\]

Абсолютная погрешность разности напряжений:
\[
	\Delta(U_{n_k} - U_{n_{k+1}}) = \Delta U_{n_k} + \Delta U_{n_{k+1}}.
\]

Абсолютная погрешность разности токов:
\[
	\Delta(I_{n_{k+1}} - I_{n_k}) = \Delta I_{n_{k+1}} + \Delta I_{n_k}.
\]

Абсолютная погрешность измерения напряжения (округляем до тысячных):
\[
	\Delta U_n = \frac{\text{цена младшего разряда}}{2} = \frac{0{,}001\,\text{В}}{2} = 0{,}0005\,\text{В}.
\]

\subsection{Пример расчёта для \( k = 2 \)}

1. \textit{Вычисляем разности напряжений и их погрешности:}
\begin{align*}
	U_{n_2} - U_{n_3}         & = 10{,}800\,\text{В} - 9{,}600\,\text{В} = 1{,}200\,\text{В},                                     \\
	\Delta(U_{n_2} - U_{n_3}) & = \Delta U_{n_2} + \Delta U_{n_3} = 0{,}0005\,\text{В} + 0{,}0005\,\text{В} = 0{,}0010\,\text{В}.
\end{align*}

2. \textit{Вычисляем токи и их погрешности:}
\begin{align*}
	I_{n_2}        & = \frac{10{,}800\,\text{В}}{5400\,\Omega} = 0{,}0020\,\text{А},                     \\
	\Delta I_{n_2} & = \frac{0{,}0005\,\text{В}}{5400\,\Omega} \approx 9{,}259 \times 10^{-8}\,\text{А}, \\
	I_{n_3}        & = \frac{9{,}600\,\text{В}}{2400\,\Omega} = 0{,}0040\,\text{А},                      \\
	\Delta I_{n_3} & = \frac{0{,}0005\,\text{В}}{2400\,\Omega} \approx 2{,}083 \times 10^{-7}\,\text{А}.
\end{align*}

3. \textit{Вычисляем разности токов и их погрешности:}
\begin{align*}
	I_{n_3} - I_{n_2}         & = 0{,}0040\,\text{А} - 0{,}0020\,\text{А} = 0{,}0020\,\text{А},                                                                             \\
	\Delta(I_{n_3} - I_{n_2}) & = \Delta I_{n_3} + \Delta I_{n_2} = 2{,}083 \times 10^{-7}\,\text{А} + 9{,}259 \times 10^{-8}\,\text{А} = 3{,}009 \times 10^{-7}\,\text{А}.
\end{align*}

4. \textit{Вычисляем \( r_2 \) и его погрешность:}
\[
	r_2 = \frac{1{,}200\,\text{В}}{0{,}0020\,\text{А}} = 600{,}000\,\Omega.
\]

\[
	\Delta r_2 = \frac{|\Delta(U_{n_2} - U_{n_3}) \cdot (I_{n_3} - I_{n_2}) + \Delta(I_{n_3} - I_{n_2}) \cdot (U_{n_2} - U_{n_3})|}{(I_{n_3} - I_{n_2})^2}.
\]

Подставляем значения:
\begin{align*}
	\Delta r_2 & = \frac{|0{,}0010\,\text{В} \cdot 0{,}0020\,\text{А} + 3{,}009 \times 10^{-7}\,\text{А} \cdot 1{,}200\,\text{В}|}{(0{,}0020\,\text{А})^2}     \\
	           & = \frac{(2{,}000 \times 10^{-6}\,\text{В}\cdot\text{А} + 3{,}6108 \times 10^{-7}\,\text{В}\cdot\text{А})}{4{,}000 \times 10^{-6}\,\text{А}^2} \\
	           & = \frac{2{,}3611 \times 10^{-6}\,\text{В}\cdot\text{А}}{4{,}000 \times 10^{-6}\,\text{А}^2} = 0{,}5903\,\Omega \approx 0{,}590\,\Omega.
\end{align*}

\subsection{Результаты расчётов}

Проводим аналогичные расчёты для всех \( k \) от 2 до 10 и получаем:

\begin{table}[H]
	\centering
	\caption*{Значения \( r_k \) и их абсолютные погрешности \( \Delta r_k \)}
	\begin{tabular}{|c|c|c|}
		\hline
		\( k \) & \( r_k \), Ом & \( \Delta r_k \), Ом \\
		\hline
		2       & 600{,}000     & \( \pm \) 0{,}590    \\
		3       & 600{,}000     & \( \pm \) 0{,}670    \\
		4       & 600{,}000     & \( \pm \) 0{,}774    \\
		5       & 600{,}000     & \( \pm \) 0{,}917    \\
		6       & 600{,}000     & \( \pm \) 1{,}125    \\
		7       & 599{,}301     & \( \pm \) 1{,}455    \\
		8       & 600{,}701     & \( \pm \) 2{,}090    \\
		9       & 602{,}015     & \( \pm \) 3{,}778    \\
		10      & 598{,}015     & \( \pm \) 2{,}711    \\
		\hline
	\end{tabular}
\end{table}

\subsection{Вычисление значения внутреннего сопротивления и его погрешности}

Среднее квадратическое значение внутреннего сопротивления:
\[
	\begin{aligned}
		r & = \sqrt{\frac{\sum\limits_{k=2}^{10} r_k^2}{9}}                                                           \\
		  & = \sqrt{\frac{(600{,}000)^2 \times 5 + (599{,}301)^2 + (600{,}701)^2 + (602{,}015)^2 + (598{,}015)^2}{9}} \\
		  & = 600{,}004\,\Omega.
	\end{aligned}
\]

Абсолютная погрешность \( \Delta r \) вычисляется как среднеквадратическое из погрешностей \( \Delta r_k \):
\[
	\Delta r = \sqrt{\frac{\sum_{k=2}^{10} (\Delta r_k)^2}{9}}.
\]

Подставляем значения:
\[
	\begin{aligned}
		\sum_{k=2}^{10} (\Delta r_k)^2 & = (0{,}59)^2 + (0{,}67)^2 + (0{,}774)^2 + (0{,}917)^2 + (1{,}125)^2 \\
		                               & \quad + (1{,}455)^2 + (2{,}09)^2 + (3{,}778)^2 + (2{,}711)^2        \\
		                               & = 0{,}3481 + 0{,}4489 + 0{,}599076 + 0{,}840889 + 1{,}265625        \\
		                               & \quad + 2{,}117025 + 4{,}3681 + 14{,}273284 + 7{,}349521            \\
		                               & = 31{,}6105.
	\end{aligned}
\]

Тогда:
\[
	\Delta r = \sqrt{\frac{31{,}6105}{9}} = \sqrt{3{,}5123} = 1{,}874\,\Omega.
\]

\subsection{Вычисление тока короткого замыкания и его погрешности}

Ток короткого замыкания:
\[
	I_{sc} = \frac{E}{r} = \frac{12{,}000\,\text{В}}{600{,}004\,\Omega} \approx 20{,}000\,\text{мА}.
\]

Абсолютная погрешность \( \Delta I_{sc} \):
\[
	\Delta I_{sc} = I_{sc} \left( \frac{\Delta E}{E} + \frac{\Delta r}{r} \right).
\]

Абсолютная погрешность ЭДС:
\[
	\Delta E = \frac{0{,}001\,\text{В}}{2} = 0{,}0005\,\text{В}.
\]

Вычисляем:
\[
	\Delta I_{sc} = 20{,}000\,\text{мА} \left( \frac{0{,}0005\,\text{В}}{12{,}000\,\text{В}} + \frac{1{,}189\,\Omega}{600{,}004\,\Omega} \right) \approx 0{,}040\,\text{мА}.
\]

Итоговое значение:
\[
	I_{sc} = (20{,}000 \pm 0{,}040)\,\text{мА}.
\]

\subsection{Результат}

Среднее внутреннее сопротивление:
\[
	r = (600{,}004 \pm 1{,}874)\,\Omega.
\]

Ток короткого замыкания:
\[
	I_{sc} = (20{,}000 \pm 0{,}040)\,\text{мА}.
\]

Экспериментальные значения совпадают с расчётными в пределах погрешности, что подтверждает корректность проведённых измерений и расчётов.


% Section 2: Таблица измерений
\subsection{Заполненная таблица 1.1}

\begin{table}[H]
	\centering
	\addtocounter{table}{0}
	\caption*{Таблица 1.1: Результаты измерений и расчётов}
	\begin{tabular}{|c|c|c|c|c|c|c|}
		\hline
		k  & \multicolumn{2}{c|}{Измерения} & \multicolumn{4}{c|}{Расчёт: r = 600.004 [Ом], E = 12 [B], Isc = 20 [мА]}                                               \\
		\hline
		0  & $R_n$ [Ом] & $U_n$ [В] & $I_n$ [мА] & $P_n$ [Вт] & $\eta$ & $r$ [Ом] \\
		\hline
		1  & $\infty$   & 12.000    & 0.00       & 0.00       & 1.0    & --       \\
		2  & 5400       & 10.800    & 2.00       & 0.022      & 0.9    & 600.00   \\
		3  & 2400       & 9.600     & 4.00       & 0.038      & 0.8    & 600.00   \\
		4  & 1400       & 8.400     & 6.00       & 0.050      & 0.7    & 600.00   \\
		5  & 900        & 7.200     & 8.00       & 0.058      & 0.6    & 600.00   \\
		6  & 600        & 6.000     & 10.00      & 0.060      & 0.5    & 600.00   \\
		7  & 400        & 4.800     & 12.00      & 0.058      & 0.4    & 599.301  \\
		8  & 257        & 3.599     & 14.004     & 0.050      & 0.3    & 600.701  \\
		9  & 150        & 2.400     & 16.00      & 0.038      & 0.2    & 602.015  \\
		10 & 67         & 1.205     & 17.985     & 0.022      & 0.1    & 598.015  \\
		11 & 0          & 0.000     & 20.00      & 0.000      & 0.0    & --       \\
		\hline
	\end{tabular}
\end{table}
