\subsection{Расчёт погрешностей}

Абсолютная погрешность измерений напряжения \( U_n \) оценивается как половина цены последнего разряда, что составляет:
\[
\Delta U_n = 0.0005 \; \text{В}.
\]

### Погрешность для тока через нагрузку \( I_n \)

Относительная погрешность тока через нагрузку \( I_n \) рассчитывается по формуле:
\[
\delta I_n = \frac{\Delta U_n}{U_n}.
\]

### Погрешность для внутреннего сопротивления \( r_k \)

Относительная погрешность для каждого значения внутреннего сопротивления \( r_k \), рассчитанного по формуле
\[
r_k = \frac{U_{n_k} - U_{n_{k+1}}}{I_{n_{k+1}} - I_{n_k}},
\]
суммируется из относительных погрешностей для разности напряжений и разности токов:
\[
\delta r_k = \sqrt{\left(\frac{\Delta (U_{n_k} - U_{n_{k+1}})}{U_{n_k} - U_{n_{k+1}}}\right)^2 + \left(\frac{\Delta (I_{n_{k+1}} - I_{n_k})}{I_{n_{k+1}} - I_{n_k}}\right)^2}.
\]
Здесь погрешность разности напряжений составляет:
\[
\Delta (U_{n_k} - U_{n_{k+1}}) = 2 \cdot \Delta U_n = 0.001 \; \text{В}.
\]
А погрешность разности токов рассчитывается через сложение погрешностей для каждого тока:
\[
\Delta (I_{n_{k+1}} - I_{n_k}) = \sqrt{\left(\frac{\Delta U_n}{R_{n_{k+1}}}\right)^2 + \left(\frac{\Delta U_n}{R_{n_k}}\right)^2}.
\]

### Погрешность для среднего квадратического сопротивления \( r \)

Среднее квадратическое сопротивление \( r \) имеет следующую погрешность:
\[
\delta r = \sqrt{\sum_{k=2}^{10} \delta r_k^2}.
\]

### Погрешность для тока короткого замыкания \( I_{sc} \)

Относительная погрешность тока короткого замыкания \( I_{sc} \) рассчитывается по следующей формуле:
\[
\delta I_{sc} = \sqrt{\left(\frac{\Delta U_0}{U_0}\right)^2 + \left(\frac{\delta r}{r}\right)^2},
\]
где \( \Delta U_0 = \Delta U_n \), так как напряжение холостого хода измеряется с той же погрешностью. Абсолютная погрешность для \( I_{sc} \) рассчитывается как произведение относительной погрешности на ток короткого замыкания:
\[
\Delta I_{sc} = I_{sc} \cdot \delta I_{sc}.
\]
